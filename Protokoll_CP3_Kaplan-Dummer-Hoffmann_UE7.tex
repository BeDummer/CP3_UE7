\documentclass[%
	paper=A4,	% stellt auf A4-Papier
	pagesize,	% gibt Papiergröße weiter
	DIV=calc,	% errechnet Satzspiegel
	smallheadings,	% kleinere Überschriften
	ngerman		% neue Rechtschreibung
]{scrartcl}
\usepackage{BenMathTemplate}
\usepackage{BenTextTemplate}

\title{{\bf Wissenschaftliches Rechnen III / CP III}\\Übungsblatt 7}
\author{Tizia Kaplan (545978)\\Benjamin Dummer (532716)\\Antoine Hoffmann (426675)\\Gruppe 10}
\date{17.06.2016}

\begin{document}
\maketitle
Online-Version: \href{https://www.github.com/BeDummer/CP3_UE7}{\url{https://www.github.com/BeDummer/CP3_UE7}}

\section*{Aufgabe 7.1}
Entsprechende Funktionen und Prozeduren wurden im Programm \url{phimain.c} implementiert und die Verifikation durchgef"uhrt. Aufgrund der Maschinengenauigkeit ist Vorsicht bei der Parameterwahl geboten, da gro\ss e Summen entstehen k"onnen, die als negatives Argument in eine Exponentialfunktion eingesetzt werden.

\section*{Aufgabe 7.2}
Die Berechnung wurde auf der CPU (\url{pi_cpu.c}) und auf der GPU (\url{pi_gpu.cu}) umgesetzt.
Bei der Implementation auf der GPU haben wir uns auf eine 1D-Struktur beschr"ankt. Dadurch ist die Anzahl der Punkte pro Durchlauf auf $1024$ begrenzt. Auf der CPU wurde ein Maximum bei $10^7$ Punkten festgestellt. Es wurden jeweils 15 Durchl"aufe gemittelt und folgende Ergebnisse gefunden:
\begin{eqnarray} \nonumber
	\begin{array}{c|c}
 \mbox{CPU} & \mbox{GPU}\\ \hline
3.14156 \pm 0.00036 & 3.118 \pm 0.047  
	\end{array}
\end{eqnarray}

\section*{Anhänge}
\begin{itemize}
	\item Datei: \url{phimain.c} (Aufgabe 1)
	\item Datei: \url{pi_cpu.c} (Aufgabe 2)
	\item Datei: \url{pi_gpu.cu} (Aufgabe 2)
\end{itemize}
\end{document}


%% Beispiel fuer Einbindung eines Bildes

%\begin{figure}
%  \centering
%  \includegraphics[width=.75\textwidth]{Dateiname}
%  \caption{Beschriftung}
%\end{figure}


%% Beispiel fuer Tabelle im Mathe-Modus

%\begin{eqnarray} \nonumber
%	\begin{array}{l|c|c}
% \mbox{Variablentyp} & \mbox{\tt int} & \mbox{\tt double}\\ \hline
% \mbox{Laufzeit [ms]} & 1.47 & 2.01  \\ 
% \mbox{Efficiency [\%]} & 100 & 100 \\
% \mbox{Throughput [GB/s]} & 47.8 & 69.4  \\
% \mbox{Occupancy} & 0.9985 & 0.9992
%	\end{array}
%\end{eqnarray}